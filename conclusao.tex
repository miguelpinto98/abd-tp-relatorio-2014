\section{Conclusão}

Neste trabalho foi essencial a análise percepção dos parâmetros de configuração do \textit{PostgreSQL}, para que se retire desempenhos cada mais superiores e aceitáveis por parte da base de dados. Ainda assim, para o \textit{benchmark} atual poderiam ainda ser otimizados mais alguns parâmetros, o \textit{default\_statistics\_target}, \textit{fsync} e \textit{commit\_delay}, por exemplo. De facto, deve-se ter consiência que a alteração dos parâmetros não se aplica a todo o tipo de situações, isto é, depende, também, do \textit{hardware} utilizado e da quantidade de carga que é gerada em determindados momentos de tempo.

Para este \textit{benchmarking} foram utilizados parâmetros de configuração relativamente bons em termos de desempenho. Apesar de nesta máquina o resultado ter sido positivo não implica que noutra máquina estes valores definidos façam com que o desempenho seja superior, daí ser sempre necessário configurar os parâmetros à medida do \textit{hardware} de determinada máquina.
