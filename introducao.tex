\section{Introdução}

Este trabalho tem como objetivo submeter uma base de dados em \textit{PostgreSQL} a vários testes de perfomance, usando uma configuração híbrida \textit{TPC-C + CH-benCHmark} com escala adequada ao hardware. Neste sentido pretende-se testar várias configurações, analisar os resultados obtidos e alcançar uma configuração que seja considerada eficiente para um determinado número de clientes.\\

Desta forma, procedeu-se a uma descrição de todo o processo, incluindo não só os testes efetuados até se alcançar a configuração ideal ou não, como também os melhoramentos que poderiam ser feitos, nomeadamente nas \textit{queries} efetuadas à base de dados.\\

\subsection{Ambiente de Testes}

De forma a que os testes corram com a menor interferência possível, todos os programas não essenciais à execução dos \textit{benchmark} foram terminados. Durante a execução dos testes teve-se o cuidado de não fazer nenhuma ação que perturbasse o desempenho e os resultados do \textit{benchmarks}.

\subsubsection{Especificações do Computador}

O computador utilizado na execução deste trabalho foi um \textit{Macbook Pro Early 2011} com as seguintes especificações:\\

\textbf{Processor:} \textit{2,3 GHz Intel Core i5}

\textbf{Memory:} \textit{8 GB 1333 MHz DDR3}

\textbf{Graphics:} \textit{Intel HD Graphics 3000 512MB}

\textbf{Software:} \textit{Samsung SSD 840 EVO 250GB}
