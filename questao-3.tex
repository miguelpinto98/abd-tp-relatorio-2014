\section{Otimização e/ou justificação do desempenho tendo em conta os parâmetros de configuração do PostgreSQL}

O objetivo desta questão passa por alterar certos parâmetros, recorrendo à configuração do \textit{PostgreSQL} para tentar aumentar a \textit{performance} da execução híbrida dos \textit{benchmarks} \verb+TPC-C+ e \verb+CH-Benchmark+.

\subsection{Configuração \textit{shared\_buffers}}

O parâmetro de configuração \textit{shared\_buffers} determina a quantidade de memória RAM dedicada ao \textit{PostgreSQL} que vai ser usada para armazenar dados enquanto executa as queries.

Inicialmente, podia-se prever que quanto maior fosse a quantidade de memória dedicada melhor seriam os resultados de desempenho. É uma conclusão precipitada, pois o \textit{PostgreSQL} utiliza esta memória para efetuar operações e não para \textit{cache} de disco, por exemplo.

Um valor razoável para para o \textit{shared\_buffers} é de 25\% da memória RAM do sistema (2GB) mas decidiu-se extender os testes até 50\% da memória do sistema (4GB). Apesar de valores acima de 40\% serem, provavelmente, menos eficazes, pois o \textit{PostgreSQL} conta também com a \textit{cache} do sistema operativo.

Para o \textit{benchamark} foram usados valores de memória RAM entre os 128MB, valor por defeito na configuração, e os 4096MB. Os resultados seguem-se na próxima tabela.

\begin{table}[!h]
\center
\small
\begin{tabular}{|c|c|c|c|c|}
\hline
\textbf{\# tamanho (MB)} & \textbf{\# pedidos} & \textbf{pedidos/s} & \textbf{lat. média (s)} & \textbf{lat. perct. 99 (s)}  \\ \hline
128 (Default) & 62118 & 310.5880666032618 & 1.0220994499090980 & 1.130365  \\ \hline
256 & 60989 & 304.9445727406344 & 1.0224922605223894 & 1.133323  \\ \hline
512 & 63196 & 315.9795274241785 & 1.0210114359611369 & 1.125195  \\ \hline
1024 & 66182 & 330.9085519342495 & 1.0217862895198089 & 1.110705  \\ \hline
2048 & 70374 & 351.8684527622802 & 1.0194107100775853 & 1.098318  \\ \hline
3072 & 64198 & 320.9880720942809 & 1.0198997528427676 & 1.129971  \\ \hline
4096 & 60497 & 302.4838197277969 & 1.0220000570937402 & 1.134960  \\ \hline
\end{tabular}
\caption{Resultados obtidos no \textit{shared\_buffers} para diferentes tamanhos de memória}
\end{table}

\begin{figure}[ht!]
\centering
\includegraphics[width=\textwidth]{img/01_sb.png}
\caption{Gráfico correspondente aos valores da tabela anterior\label{overflow}}
\end{figure}

Ao analisarmos os resultados obtidos, verifica-se que o percentil da latência média e a 99\% estão constantemente a diminuir à medida que a quantidade de memória disponível aumenta. Isto acontece devido à ausência de escrita dos resultados para o disco, o que provoca a redução na quantidade de \textit{I/O} para a obtenção de dados. Conclui-se ainda que o melhor resultado aconteceu quando a memória disponível era de \textbf{2048MB}, tendo ocorrido mais pedidos por segundo e as latências (média e 99\%) apresentarem os resultados mais inferiores. A segunda melhor configuração para esta métrica aconteceu quando a memória disponível era de \textbf{1024MB}.

Assim sendo, para esta configuração o valor escolhido será o de \textbf{2048MB} por apresentar os melhores resultados tanto a nível de pedidos como de latência.

\subsection{Configuração \textit{effective\_cache\_size}}

O \textit{effective\_cache\_size} é definido como uma estimativa da quantidade de memória disponível para a \textit{cache} do disco. Este parâmetro é utilizado \textit{PostgreSQL Query Planner} para perceber que planos pode utilizar tendo em conta o espaço disponível em memória.

A documentação do \textit{PostgreSQL} afirma que definir 50\% da memória total do dispositivo será uma configuração conservadora e que 75\% uma configuração agressiva, mas aceitável.

Então para a configuração deste \textit{benchmark} utilizou-se os seguintes parâmetros, 128MB, 512MB, 1024MB, 2048MB, 4096MB e 6144MB.

\begin{table}[!h]
\center
\small
\begin{tabular}{|c|c|c|c|c|}
\hline
\textbf{\# tamanho (MB)} & \textbf{\# pedidos} & \textbf{pedidos/s} & \textbf{lat. média (s)} & \textbf{lat. perct. 99 (s)}  \\ \hline
128 (Default) & 62118 & 310.58806660326184 & 1.022099449909098 & 1.130365  \\ \hline
512 & 67022 & 335.1094363325237 & 1.0208270027602877 & 1.112753  \\ \hline
1024 & 66987 & 334.93426967582496 & 1.0204846512009793 & 1.109516  \\ \hline
2048 & 66478 & 332.38835629969145 & 1.020474478669635 & 1.112206  \\ \hline
4096 & 59880 & 299.3983502432346 & 1.0226386898630595 & 1.131897  \\ \hline
6144 & 63982 & 319.9084020175434 & 1.0210112850958082 & 1.130562  \\ \hline
\end{tabular}
\caption{Resultados obtidos no \textit{effective\_cache\_size} para diferentes tamanhos de memória}
\end{table}

\begin{figure}[ht!]
\centering
\includegraphics[width=70mm]{img/02_ecs_b.png}
\includegraphics[width=70mm]{img/02_ecs_a.png}
\caption{Gráfico correspondente aos valores da tabela anterior\label{overflow}}
\end{figure}


Ao analisar-se os valores obtidos identificasse que na configuração mais conservadora (\textbf{4096MB}) e agressiva (\textbf{6144MB}) não se obteve os melhores resultados. Pensámos que se deve ao facto de a base de dados do benchmark não ter um tamanho, cerca de 650MB, que faça a diferença neste aspecto. O número de pedidos medidos foi muito inferior aos outros valores assim como o percentil da latência a 99\%, a latência média, praticamente, encontra-se entre a média dos resultados.

Podemos, então, definir o intervalo de \textbf{[512-2048]}MB como o que se obteve melhor resultados para esta configuração. O valor \textbf{1024MB} obteve melhor resultados no percentil da lantência a 99\%. Já o valor \textbf{512MB} obteve mais pedidos por segundo e um ligeiro aumento nas latências no teste, este último influência negativamente o resultado. Com o parâmetro a \textbf{2048MB} consegue-se manter na média dos resultados anteriores, possuíndo o resultado com menor latência média.

Dentro do intervalo definido anteriormente se fosse necessário escolher um resultado como melhor valor, escolhíamos a configuração com \textbf{1024MB} de \textit{effective\_cache\_size} onde se obteve a menor latência a 99\% e valor similares na latência média para o mesmo número de pedidos.

\newpage

\subsection{Configuração \textit{checkpoint\_segments}}

O \textit{PostgreSQL} escreve novas transações na base de dados em ficheiros chamados de segmentos \textit{Write-Ahead Logging (WAL)} que tem um tamanho de 16MB. Nas configurações do \textit{postgresql.conf} o valor atualmente por defeito é de 3 \textit{checkpoint\_segments}.

Aumentar o número de \textit{checkpoint\_segments} melhora o desempenho do benchmark pois faz com que os \textit{checkpoints} ocorram com menor frequência, obrigando assim a base de dados a uma recuperação mais lenta após uma falha. No entanto o \textit{I/O} ao disco diminui provocando melhorias, eventuais, no teste.

Para este \textit{benchmarking} foram utilizados os seguintes valores de \textit{checkpoint\_segments}: 3, 9 e 16.

\begin{table}[!h]
\center
\small
\begin{tabular}{|c|c|c|c|c|}
\hline
\textbf{\# segmentos} & \textbf{\# pedidos} & \textbf{pedidos/s} & \textbf{lat. média (s)} & \textbf{lat. perct. 99 (s)}  \\ \hline
3 (Default) & 62118 & 310.58806660326184 & 1.022099449909098 & 1.130365  \\ \hline
9 & 66254 & 331.26961583822094 & 1.0211749849065717 & 1.113339  \\ \hline
16 & 54564 & 272.81969749069896 & 1.0238594005571438 & 1.103626  \\ \hline
\end{tabular}
\caption{Resultados obtidos para o \textit{checkpoint\_segments}}
\end{table}

\begin{figure}[ht!]
\centering
\includegraphics[width=70mm]{img/03_cs_a.png}
\includegraphics[width=70mm]{img/03_cs_b.png}
\caption{Gráfico correspondente aos valores da tabela anterior\label{overflow}}
\end{figure}

Como foi de esperar, ao aumentar o valor de \textit{checkpoint\_segments} o valores de transações aumentaram, mas somente quando o valor foi de \textbf{9} segmentos. Quando o teste foi efetuado com \textbf{16} segmentos o número de transações diminuiu abruptamente, mas na latência a 99\% obteve-se o melhor resultado, ainda que a latência média se tenha mantido. Estes resultados devem-se à recuperação mais lenta da base de dados.

Nesta situação, a configuração mais indicada seria de \textbf{9} segmentos. O \textit{benchmark} obteve o melhor resultado ao nível das transações e de latência média, que foi inferior.

Normalmente para valores elevados de \textit{checkpoint\_segments} é recomendado aumentar o \textit{checkpoint\_timeout} por causa dos intervalos de controlo.

\newpage

\subsection{Configuração \textit{autovacuum\_naptime}}

Este processo corresponde ao tempo mínimo entre a execução do \textit{autovacuum} (limpar resíduos deixados na base de dados), isto é, só ocorre um novo depois de ter passado o tempo de execução do anterior.

Para se perceber o que acontece nesta configuração foram feitos vários \textit{benchmarks} para os seguintes tempos de \textit{autovacuum\_naptime}: desligado, 1 minuto, 2 minutos e 3 minutos entre cada execução. Obteve-se, assim, os resultados da próxima tabela.

\begin{table}[!h]
\center
\small
\begin{tabular}{|c|c|c|c|c|}
\hline
\textbf{\# tempo (min)} & \textbf{\# pedidos} & \textbf{pedidos/s} & \textbf{lat. média (s)} & \textbf{lat. perct. 99 (s)}  \\ \hline
0 (off) & 59802 & 299.0081276096099 & 1.021858928530818 & 1.136963  \\ \hline
1 (Default) & 62118 & 310.5880666032618 & 1.022099449909098 & 1.130365  \\ \hline
2 & 53296 & 266.4799488385146 & 1.025329163370609 & 1.138289  \\ \hline
3 & 61163 & 305.8148124055486 & 1.022736959518009 & 1.124206  \\ \hline
\end{tabular}
\caption{Resultados obtidos para a configuração do \textit{autovacuum\_naptime}}
\end{table}

\begin{figure}[ht!]
\centering
\includegraphics[width=70mm]{img/05_vacuum_a.png}
\includegraphics[width=70mm]{img/05_vacuum_b.png}
\caption{Gráficos correspondente aos valores da tabela \label{overflow}}
\end{figure}

Dos resultados apresentados verifica-se que aqueles que apresentam melhores tempos de execução foram o de \textbf{1} minuto e \textbf{3} minutos. Este último só ocorreu no máximo uma vez, assim como o teste a \textbf{2} minutos, o que leva a querer que não teve muita influência nos valores da latência. Com o parâmetro desligado não se obteve os resultados mais corretos, isto porque não ocorreu nenhuma execução do \textit{\textbf{autovacuum}} o que contribui para estatísticas erradas e consequentemente para escolha errada dos algoritmos usados pelo \textit{PostgreSQL}.

Ao contrário do esperado ter um \textit{\textbf{autovacuum}} mais regular faz com que os resultados obtidos fossem mais razoáveis. De facto, ao haver com mais frequência o \textit{\textbf{autovacuum}} faz com que em cada operação haja menos dados para limpar.

Assim, decidiu-se escolher para o \textit{autovacuum\_naptime} como melhor resultado o tempo de \textbf{1} minuto, por defeito na configuração do \textit{postgresql.conf}.

\newpage

\subsection{Configuração \textit{work\_mem}}

O \textit{work\_mem} especifica a quantidade de memória RAM que pode ser usada para operações internas (\textit{Sort e Hash Tables}) antes de serem gravadas para ficheiros temporários em disco. O valor predefinido na configuração do \textit{PostgreSQL} é de 1MB.

Como existem algumas \textit{queries} complexas no teste do \textit{CHBenchmark} decidimos testar como vão variar o tempo de execução das mesmas e perceber se é possível obter mais transações com a menor latência possível. Para isto foram feitos testes para 1MB, 8MB, 32MB, 128MB e 512MB.

\begin{table}[!h]
\center
\small
\begin{tabular}{|c|c|c|c|c|}
\hline
\textbf{\# tamanho (MB)} & \textbf{\# pedidos} & \textbf{pedidos/s} & \textbf{lat. média (s)} & \textbf{lat. perct. 99 (s)}  \\ \hline
1 (Default) & 62118 & 310.5880666032618 & 1.022099449909098 & 1.130365  \\ \hline
8 & 62391 & 311.9513589177765 & 1.021819045423218 & 1.130889  \\ \hline
32 & 67109 & 335.5443187426919 & 1.019860019833405 & 1.117067  \\ \hline
128 & 67728 & 338.6382522575026 & 1.018854204671627 & 1.118497  \\ \hline
512 & 67698 & 338.4895027961542 & 1.020378764261869 & 1.118507  \\ \hline
\end{tabular}
\caption{Resultados obtidos no \textit{work\_mem} para diferentes tamanhos de memória}
\end{table}

\begin{figure}[ht!]
\centering
\includegraphics[width=70mm]{img/06_wm_a.png}
\includegraphics[width=70mm]{img/06_wm_b.png}
\caption{Gráficos correspondente aos valores da tabela \label{overflow}}
\end{figure}

Através das ferramentas que acompanham o \textit{oltpbench}, nomeadamente o \textit{plot\_latencies.py} obtivemos o seguinte gráfico, para os valores médios de execução:

\begin{figure}[ht!]
\centering
\includegraphics[width=\textwidth]{img/06_wm_c.png}
\caption{Gráfico com os valores de execução para cada query com diferentes valores de memória\label{overflow}}
\end{figure}

Como verificámos no gráfico anterior o tempo de execução variou significativamente em duas \textit{queries}, na \textit{Query 10} que tinha um tempo de execução superior na configuração inicial e que passou a obter um tempo de execução inferior, proporcionalmente à memória disponível. Já na \textit{Query 5} que inicialmente tinha um tempo de execução inferior passou a tempos muitos superiores ao esperado. Isto deve-se ao facto dos algoritmos escolhidos terem efeito na quantidade de memória definida.

Nos resultados das \textit{queries 1, 6 e 17} não se surtiu grandes variações.

Caso fosse necessário escolher a melhor configuração para este parâmetro, este situava-se entre os \textbf{8MB e 32MB}. Será necessário efetuar mais um teste, por exemplo, com \textbf{16MB} para perceber se houve alguma alteração na execução das \textit{queries} que tire partido da memória disponível.

Finalmente, se analisarmos o valor das latências junto com as transações medidas, percebe-se que \textbf{32MB} será provavelmente uma boa configuração para este parâmetro.


\subsection{Configuração do \textit{random\_page\_cost}}

Como o dispositivo utilizado para correr os testes possui um \textit{Solid State Drive (SSD)} decidou-se alterar na configuração do \textit{PostgreSQL} o custo dos acessos aleatórios. Neste teste vamos mostrar os resultados obtidos para a configuração inicial que tem definido um custo igual a 4, e alterações de custo entre 1 e 2.

\begin{table}[!h]
\center
\small
\begin{tabular}{|c|c|c|c|c|}
\hline
\textbf{\# custo} & \textbf{\# pedidos} & \textbf{pedidos/s} & \textbf{lat. média (s)} & \textbf{lat. perct. 99 (s)}  \\ \hline
4.0 (Default) & 62118 & 310.58806660326184 & 1.022099449909098 & 1.130365  \\ \hline
2.0 & 65295 & 326.47479814226466 & 1.0193406799754958 & 1.116984  \\ \hline
1.0 & 58540 & 292.69854964502616 & 1.0225425792791254 & 1.133753  \\ \hline
\end{tabular}
\caption{Resultados obtidos para 3 tipos diferentes de custos}
\end{table}

\begin{figure}[ht!]
\centering
\includegraphics[width=70mm]{img/07_rpc_a.png}
\includegraphics[width=70mm]{img/07_rpc_b.png}
\caption{Gráficos correspondente aos valores da tabela \label{overflow}}
\end{figure}

Na análise dos resultados obtidos nas tabelas e gráficos anteriores verificasse que se o custo dos acessos aleatórios for de \textbf{2} encontram-se melhores valores de latência média e a 99\%. Também o número de pedidos médios por segundo é superior.

De notar, que quando o custo é de \textbf{1} os valores são muito inferiores aos desejados.

Conclui-se então que se diminuir o custo dos acessos aleatórios para \textbf{2} irá beneficiar a execução do \textit{benchmark}. Isto acontece porque a escolha dos algoritmos vai passar a optar em algumas partes da \textit{query} por usar acessos aleatórios em vez dos, preteridos, acessos sequenciais.

\newpage

\subsection{Configuração do \textit{shared\_buffers} com o \textit{effective\_cache\_size}}

Escolhida a melhor configuração no \textit{shared\_buffers}, o intervalo [1024-2048]MB de memória. No \textit{effective\_cache\_size} decidiu-se usar o melhor resultado da configuração anterior, 1024MB, e testar, também, os valores de memória conservadora e agressiva definidos anteriormente, 4096MB e 6144MB, respectivamente.

Os resultados obtidos para esta configuração são os seguintes:

\begin{table}[!h]
\center
\small
\begin{tabular}{|c|c|c|c|c|}
\hline
\textbf{\# tamanho} & \textbf{\# pedidos} & \textbf{pedidos/s} & \textbf{lat. perct. média} & \textbf{lat. perct. 99}  \\ \hline
1024MB & 69015 & 345.073418193803 & 1.02032663525321 & 1.094924  \\ \hline
4096MB & 67377 & 336.882924828134 & 1.02042478753877 & 1.095611  \\ \hline
6144MB & 65443 & 327.214804387718 & 1.02116393962685 & 1.092516  \\ \hline
\end{tabular}
\caption{Resultados obtidos para o valor de 1024MB de \textit{shared\_buffers}}
\end{table}

\begin{table}[!h]
\center
\small
\begin{tabular}{|c|c|c|c|c|}
\hline
\textbf{\# tamanho} & \textbf{\# pedidos} & \textbf{pedidos/s} & \textbf{lat. perct. média} & \textbf{lat. perct. 99}  \\ \hline
1024MB & 74724 & 373.619008187245 & 1.01803675496494 & 1.088794  \\ \hline
4096MB & 66985 & 334.923957423586 & 1.01998354016571 & 1.112503  \\ \hline
6144MB & 69708 & 348.538465057512 & 1.01924645791014 & 1.095748  \\ \hline
\end{tabular}
\caption{Resultados obtidos para o valor de 2048MB de \textit{shared\_buffers}}
\end{table}

\begin{figure}[ht!]
\centering
\includegraphics[width=70mm]{img/questao_3/sb_ecs_a.png}
\includegraphics[width=70mm]{img/questao_3/sb_ecs_b.png}
\includegraphics[width=70mm]{img/questao_3/sb_ecs_c.png}
\includegraphics[width=70mm]{img/questao_3/sb_ecs_d.png}
\caption{Gráficos corresponde às variações do \textit{shared\_buffers} para cada \textit{effective\_cache\_size}}
\end{figure}

Desta vez, conseguiu-se perceber que a configuração mais agressiva do \textit{effective\_cache\_size} (\textbf{6144MB}) teve melhores resultados dos que nos resultados indivuais do parâmetro.

Conclui-se facilmente que a melhor configuração obtida foi de \textbf{2048MB} de \textit{shared\_buffers} e de \textbf{1024MB} de \textit{effective\_cache\_size}. Através da análise dos gráficos confirmasse que se obteve mais pedidos por segundo e o percentil das latências (média e a 99\%) são os mais baisxos. Esta configuração será mantida para a próxima configuração.

Desta forma, percebe-se que ter muita quantidade de memória RAM disponível não significa que se vai obter resultados com qualidade superior, antes pelo contrário, como provado neste \textit{benchmark}.

\subsection{Configuração anterior com o \textit{checkpoint\_segments}}

\begin{table}[!h]
\center
\small
\begin{tabular}{|l|c|}
\hline
\textbf{Parâmetro} & \textbf{Valor} \\ \hline
\textbf{shared\_buffers} & 2048MB  \\ \hline
\textbf{effective\_cache\_size} & 1024MB  \\ \hline
\end{tabular}
\end{table}

Para os melhores resultados obtidos na configuração individual do \textit{checkpoint\_segments}, 3 e 9 segmentos, vamos testar estes valores com a configuração definida na tabela anterior.

\begin{table}[!h]
\center
\small
\begin{tabular}{|c|c|c|c|c|}
\hline
\textbf{\# segmentos} & \textbf{\# pedidos} & \textbf{pedidos/s} & \textbf{lat. média (s)} & \textbf{lat. perct. 99 (s)}  \\ \hline
3Segmentos & 74724 & 373.619008187245 & 1.01803675496494 & 1.088794  \\ \hline
9Segmentos & 75410 & 377.048437684716 & 1.01657169486805 & 1.085938  \\ \hline
\end{tabular}
\caption{Resultados com variações do número de segmentos}
\end{table}

\begin{figure}[ht!]
\centering
\includegraphics[width=70mm]{img/questao_3/sb_ecs_cs_a.png}
\includegraphics[width=70mm]{img/questao_3/sb_ecs_cs_b.png}
\includegraphics[width=70mm]{img/questao_3/sb_ecs_cs_c.png}
\includegraphics[width=70mm]{img/questao_3/sb_ecs_cs_d.png}
\caption{Gráficos corresponde às variações das configurações anteriores com \textit{checkpoint\_segments}}
\end{figure}

Como esperado o aumento do número de \textit{checkpoint\_segments} permitiu aumentar o número de transações medidas e diminuir as latências (média e a 99\%).

Assim para a próxima configuração será usado um \textit{checkpoint\_segments} de 9 segmentos. De notar que apesar deste aumento significativo dos resultados o tempo de recuperação da Base de Dados será mais lento e que ainda que o número de segmentos tenha aumentado ainda é possível notar que em determinados momentos os \textit{checkpoints} são ainda muito frequentes. Como tal, poderia-se aumentar o número de segmentos, mas poderia causar alguns problemas de desempenho no \textit{benchmark}.

\newpage

\subsection{Configuração anterior com o \textit{autovacuum\_naptime}}

\begin{table}[!h]
\center
\small
\begin{tabular}{|l|c|}
\hline
\textbf{Parâmetro} & \textbf{Valor} \\ \hline
\textbf{shared\_buffers} & 2048MB  \\ \hline
\textbf{effective\_cache\_size} & 1024MB  \\ \hline
\textbf{checkpoint\_segments} & 9 Segmentos \\ \hline
\end{tabular}
\end{table}

Como na configuração individual do \textit{autovacuum\_naptime} não se obteve resultados razoáveis para 2 e 3 minutos, decidiu-se para este teste efetuar novos tempos de execução do \textit{autovacuum}. Assim vamos usar 30, 60 (predefinido) e 90 segundos para a configuração deste \textit{benchmark}.

Com os valores definidos na tabela anterior mais os novos parâmetros obteve-se os seguintes resultados, representados na seguinte tabela e nos próximos gráficos.

\begin{table}[!h]
\center
\small
\begin{tabular}{|c|c|c|c|c|}
\hline
\textbf{\# tempos} & \textbf{\# pedidos} & \textbf{pedidos/s} & \textbf{lat. média (s)} & \textbf{lat. perct. 99 (s)}  \\ \hline
30Segundos & 76469 & 382.342824656676 & 1.01569113987367 & 1.088295  \\ \hline
60Segundos & 75410 & 377.048437684716 & 1.01657169486805 & 1.085938  \\ \hline
90Segundos & 74100 & 370.498798459429 & 1.0184869048448 & 1.089411  \\ \hline
\end{tabular}
\caption{Resultados obtidos para diferentes tempos de \textit{autovacuum\_naptime}}
\end{table}

\begin{figure}[ht!]
\centering
\includegraphics[width=70mm]{img/questao_3/sb_ecs_cs_vacuum_a.png}
\includegraphics[width=70mm]{img/questao_3/sb_ecs_cs_vacuum_b.png}
\includegraphics[width=70mm]{img/questao_3/sb_ecs_cs_vacuum_c.png}
\includegraphics[width=70mm]{img/questao_3/sb_ecs_cs_vacuum_d.png}
\caption{Corresponde às variações das configurações anteriores com \textit{autovacuum\_naptime}}
\end{figure}

A partir dos resultados acima mostrados somos capazes de verificar que o melhor resultado aconteceu quando o tempo de \textbf{30} segundos estava definido. Com o aumento do valor do \textit{autovacuum\_naptime} nota-se que os resultados pioraram de forma significativa.

Como tal na próxima configuração o valor do \textit{autovacuum\_naptime} será de \textbf{30} segundos.

\newpage

\subsection{Configuração anterior com o \textit{random\_page\_cost}}

\begin{table}[!h]
\center
\small
\begin{tabular}{|l|c|}
\hline
\textbf{Parâmetro} & \textbf{Valor} \\ \hline
\textbf{shared\_buffers} & 2048MB  \\ \hline
\textbf{effective\_cache\_size} & 1024MB  \\ \hline
\textbf{checkpoint\_segments} & 9 Segmentos \\ \hline
\textbf{autovacuum\_naptime} & 30 Segundos \\ \hline
\end{tabular}
\end{table}

Como na configuração individual deste parâmetro o melhor resultado obtido foi um custo de 2, decidiu-se para este novo \textit{benchmark} fazer usar este valor e também tornar novamente iguais os valores das acessos sequenciais e das acessos aleatórias, de forma a perceber como iriam variar os resultados.

\begin{table}[!h]
\center
\small
\begin{tabular}{|c|c|c|c|c|}
\hline
\textbf{\# custo} & \textbf{\# pedidos} & \textbf{pedidos/s} & \textbf{lat. média (s)} & \textbf{lat. perct. 99 (s)}  \\ \hline
Custo1 & 69864 & 349.318453834434 & 1.01832794880053 & 1.091778  \\ \hline
Custo2 & 77165 & 385.822548996668 & 1.01697276913108 & 1.085727  \\ \hline
Custo4 & 76469 & 382.342824656676 & 1.01569113987367 & 1.088295  \\ \hline
\end{tabular}
\caption{Resultados obtidos para diferentes custos do \textit{random\_page\_cost}}
\end{table}

\begin{figure}[ht!]
\centering
\includegraphics[width=70mm]{img/questao_3/sb_ecs_cs_vacuum_rpc_a.png}
\includegraphics[width=70mm]{img/questao_3/sb_ecs_cs_vacuum_rpc_b.png}
\includegraphics[width=70mm]{img/questao_3/sb_ecs_cs_vacuum_rpc_c.png}
\includegraphics[width=70mm]{img/questao_3/sb_ecs_cs_vacuum_rpc_d.png}
\caption{Corresponde às variações das configurações anteriores com \textit{random\_page\_cost}}
\end{figure}

Da análise dos valores resultantes da tabela e dos gráficos anteriores, verifica-se, novamente, que o custo \textbf{2} apresenta os melhores resultados, tanto ao nível dos pedidos como das latências. Já o custo \textbf{1} reproduz o pior resultado. Com este valor mostra que o \textit{PostgreSQL} altera os tipos de algoritmos escolhidos, passando de preterir alguns índices sequenciais a aleatórios fazendo atingir valores inviavéis para a configuração.

Com o custo igual a \textbf{2} no \textit{random\_page\_cost}, significa que é uma configuração equilibrada na escolha de determinados algoritmos.

\newpage

\subsection{Configuração anterior com o \textit{work\_mem}}

\begin{table}[!h]
\center
\small
\begin{tabular}{|l|c|}
\hline
\textbf{Parâmetro} & \textbf{Valor} \\ \hline
\textbf{shared\_buffers} & 2048MB  \\ \hline
\textbf{effective\_cache\_size} & 1024MB  \\ \hline
\textbf{checkpoint\_segments} & 9 Segmentos \\ \hline
\textbf{autovacuum\_naptime} & 30 Segundos \\ \hline
\textbf{random\_page\_cost} & 2 \\ \hline
\end{tabular}
\end{table}

Anteriormente, o melhor resultado para este parâmetro situava-se no intervalo [8-32]MB. Para este teste não seguímos diretamente esse intervalo. Foram realizados testes para 16MB, 32MB e 64MB, isto aconteceu, porque se percebeu que poderíamos atingir um resultado melhor para esta configuração.

Seguem-se então os resultados obtidos:

\begin{table}[!h]
\center
\small
\begin{tabular}{|c|c|c|c|c|}
\hline
\textbf{\# tamanho} & \textbf{\# pedidos} & \textbf{pedidos/s} & \textbf{lat. média (s)} & \textbf{lat. perct. 99 (s)}  \\ \hline
1MB & 77165 & 385.822548996668 & 1.01697276913108 & 1.085727  \\ \hline
16MB & 78158 & 390.788969075252 & 1.0171380589447 & 1.085266  \\ \hline
32MB & 79480 & 397.398960072492 & 1.01716957381731 & 1.087713  \\ \hline
64MB & 80278 & 401.388611052913 & 1.01766492785072 & 1.090232  \\ \hline
\end{tabular}
\caption{Resultados obtidos para diferentes tamanhos do \textit{work\_mem}}
\end{table}

\begin{figure}[ht!]
\centering
\includegraphics[width=70mm]{img/questao_3/sb_ecs_cs_vacuum_rpc_wm_a.png}
\includegraphics[width=70mm]{img/questao_3/sb_ecs_cs_vacuum_rpc_wm_b.png}
\includegraphics[width=70mm]{img/questao_3/sb_ecs_cs_vacuum_rpc_wm_c.png}
\includegraphics[width=70mm]{img/questao_3/sb_ecs_cs_vacuum_rpc_wm_d.png}
\caption{Corresponde às variações das configurações anteriores com \textit{work\_mem}}
\end{figure}

Nos resultados gerados verificámos que para \textbf{1MB} as latências têm valores inferiores às restantes no entanto o número de transações é inferior. Os tamanhos \textbf{16MB}, \textbf{32MB} e \textbf{64MB} apresentam um valor significativamente mais elevado de latência, mas conseguem processar mais transações por segundo.

Escolhemos o valor de \textbf{64MB} como o melhor resultado deste \textit{benchmark} porque apesar da latência aumentar significativamente, o nível de pedidos por segundos é muito superior. De notar, na imagem mais abaixo, que o tempo de execução de cada \textit{query}, indivualmente, diminuiu o seu tempo de execução.

\subsection{Configuração final}

\begin{table}[!h]
\center
\small
\begin{tabular}{|l|c|}
\hline
\textbf{Parâmetro} & \textbf{Valor} \\ \hline
\textbf{shared\_buffers} & 2048MB  \\ \hline
\textbf{effective\_cache\_size} & 1024MB  \\ \hline
\textbf{checkpoint\_segments} & 9 Segmentos \\ \hline
\textbf{autovacuum\_naptime} & 30 Segundos \\ \hline
\textbf{random\_page\_cost} & 2 \\ \hline
\textbf{work\_mem} & 64MB \\ \hline
\end{tabular}
\caption{Valores recomendados para a configuração do \textit{PostgreSQL}}
\end{table}

Agora é possível comparar os resultados entre a configuração incialmente usada, a predefinida no \textit{PostgrSQL}, e a que conseguímos obter por alteração de alguns critérios. Assim na próxima tabela vai ser possível verificar estes resultados.

\begin{table}[!h]
\center
\small
\begin{tabular}{|c|c|c|c|c|}
\hline
\textbf{\# configuração} & \textbf{\# pedidos} & \textbf{pedidos/s} & \textbf{lat. média (s)} & \textbf{lat. perct. 99 (s)}  \\ \hline
Default & 62118 & 310.5880666032618 & 1.0220994499090980 & 1.130365  \\ \hline
Final & 80278 & 401.388611052913 & 1.01766492785072 & 1.090232  \\ \hline
\end{tabular}
\caption{Configuração inicial vs. Configuração final}
\end{table}

Conclui-se que as evoluções que tem vindo a ocorrer com a alteração de certos parâmetros tem feito evoluir positivamente o desempenho do \textit{benchmark}. Notasse que o número de transações por segundo aumentou e que a latência entre estas transações diminuiu significativamente.

Segue-se ainda um gráfico que mostra o tempo que cada \textit{query} em média demorou a executar e o número de pedidos por segundo.

\begin{figure}[ht!]
\centering
\includegraphics[width=70mm]{img/questao_3/figure_1.png}
\includegraphics[width=70mm]{img/questao_3/figure_2.png}
\caption{Tempo de execução de cada \textit{query} e o número de pedidos/s}
\end{figure}
